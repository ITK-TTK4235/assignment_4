\section{Oppgave - Grunnleggende debugging med \texttt{gdb}}\label{sec:3-oppgave}

Denne øvingen er strukturert som en walkthrough. Gjennom øvingen skal dere kompilere feilaktige kodesnutter og debugge den med \verb|gdb|. Ofte vil feilene være veldig opplagt, men motstå allikevel fristelsen til å skippe over. Hensikten med denne øvingen, er at dere skal vite hvordan dere bruker GDB, så for å få godkjent, må dere vise en studass at dere til en viss grad forstår hva som skjer i denne øvingen. Terskelen er ikke altfor høy, og dere får ganske greit godkjent om dere viser at dere forstår hvordan dere kan debugge et vilkårlig program med GDB.

\subsection{Breakpoints}

Breakpoints er den mest brukte-, og en av de nyttigste egenskapene til en debugger. De tillater brukeren å midlertidig stoppe kodekjøringen for å inspisere variabler, og liste opp hvordan man kom til linjen man stoppet på. GDB operer med tre forskjellige typer breakpoints:

\begin{itemize}
    \item \textit{Breakpoint} stopper kodekjøringen når man kommer til en spesifikk linje.
    \item \textit{Watchpoint} stopper kodekjøringen dersom en spesifisert variabel eller minneadresse endrer verdi.
    \item \textit{Catchpoint} stopper kodekjøringen dersom en definert hendelse inntreffer. Hendelsen kan for eksempel være et kall til \verb|exec|, et \verb|syscall|, en \verb|assert|, eller en \verb|exception|.
\end{itemize}

Av disse er vanlig \textit{breakpoints} desidert av mest nytte. Selv om \textit{watch}- og \textit{catchpoints} av og til kan være praktiske å ha, blir de fort ubrukelige dersom man debugger kode med flere tråder.

For å demonstrere hvordan man vanligvis bruker \textit{breakpoints}, skal vi bruke programmet \verb|break|. Kall \verb|make break| for å kompilere, og \verb|gdb -tui break| (i samme mappe som programmet \verb|break|) for å starte GDB med programmet vårt. 

\textit{Breakpoints} kan settes på forskjellige måter. Om man nå skriver \verb|break local_call| setter man et \textit{breakpoint} på starten av funksjonen \verb|local_call|. Dersom man nå kaller \verb|break 9| setter man et \textit{breakpoint} på kodelinje 9. Gitt at du nå har gjort alt dette, kan vi kalle \verb|run| og få noe tilsvarende som dette:

\begin{lstlisting}[mathescape=true,keywordstyle=\color{black}]
(gdb) break local_call
Breakpoint 1 at 0x63e: file source/break.c, line 4.
(gdb) break 9
Breakpoint 2 at 0x655: file source/break.c, line 11.
(gdb) run
Starting program: /home/student/Desktop/Øving - 4/programs/break
Breakpoint 2, main () at source/break.c:11
\end{lstlisting}


Som man kan se, så stopper GDB først på linje 11, til tross for at vi satte et \textit{breakpoint} til linje 9. Dette er fordi det ikke skjer noe spennende på linje 9; vi oppretter kun en variabel, men vi gir den ikke noen verdi. GDB operer egentlig ikke med kodelinjene du ser på toppen av skjermen, men med maskininstruksjonene de har kompilert til. Det kan derfor fint hende at linje 9 ikke svarer til en instruksjon GDB kan stoppe ved (i dette eksemplet så ser man at GDB automatisk setter en \textit{breakpoint} i linje 11). Dette er verdt å ha i bakhodet dersom GDB tilsynelatende ikke gjør det du sier.

I tillegg til vanlig \textit{breakpoints} har GDB også betingede \textit{breakpoints}. Dette er \textit{breakpoints} som kun aktiveres dersom en betingelse er oppfylt. Skriv \verb|break 14 if| \verb|i>60| for å sette et \textit{breakpoint} midt i \verb|for|-løkken, som kun aktiveres dersom teller-variabelen er større enn 60. Hvis man kaller \verb|continue| for å gå videre fra linje 11, vil dere stoppe på linje 14. Om dere nå bruker \verb|print|-kommandoen vil dere se at teller-variabelen \verb|i| har verdi 61. Altså ble ikke \textit{breakpointet} aktivert før vi ønsket.

Sett derimot nå at vi i utgangspunktet tok feil; vi ønsket ikke å stoppe når \verb|i>60|, men \verb|i>80|. Om vi skriver \verb|info break| får vi opp en oversikt over hvilke \textit{breakpoints} vi har:

\begin{lstlisting}[mathescape=true,keywordstyle=\color{black}]
(gdb) info break
Num    Type       Disp    Enb     Address             What
1      breakpoint keep    y       0x000055555555463e  in local_call at
-> source/break.c:4
2      breakpoint keep    y       0x0000555555554655  in main at
->source/break.c:11
       breakpoint already hit 1 time
3      breakpoint keep    y       0x0000555555554665  in main at
->source/break.c:14
       stop only if i > 60
       breakpoint already hit 1 time
\end{lstlisting}

Vi kan nå kalle \verb|condition 3 i > 80| for å endre betingelsen til \textit{breakpoint} nummer 3 til å kunne aktiveres dersom \verb|i| er større enn 80. Nå kan dere kalle \verb|continue|, etterfulgt av \verb|print i|, som vil demonstrere at vi faktisk bare stopper når vi vil:

\begin{lstlisting}[mathescape=true,keywordstyle=\color{black}]
(gdb) condition 3 i > 80
(gdb) continue
Continuing.
Breakpoint 3, main () at source/break.c:14
(gdb) print i
$2 = 81
\end{lstlisting}

Vi kan fjerne betingelsen fra \textit{breakpoint} 3 ved å simpelthen kalle \verb|condition 3|. Om vi nå kaller \verb|continue| vil vi derimot stoppe i \verb|for|-løkken hver gang, fordi \verb|i| nå alltid er større enn 80. Om vi er ferdig med \textit{breakpoint} 3, kan vi skru det av ved å kalle \verb|disable 3| - dette vil endre \verb|Enb|-feltet (Enable) fra \verb|y| til \verb|n|:

\begin{lstlisting}[mathescape=true,keywordstyle=\color{black}]
(gdb) disable 3
(gdb) info break
Num    Type       Disp    Enb     Address             What
1      breakpoint keep    y       0x000055555555463e  in local_call at
->source/break.c:4
2      breakpoint keep    y       0x0000555555554655  in main at
->source/break.c:11
       breakpoint already hit 1 time
3      breakpoint keep    n       0x0000555555554665  in main at
->source/break.c:14
       breakpoint already hit 3 times|
\end{lstlisting}

Vi kan også permanent slette \textit{breakpoint} 3 ved å kalle \verb|delete 3|, om vi vet at vi ikke har behov for det igjen. I tillegg til å jobbe på en fil, kan GDB også sette \textit{breakpoints} i andre filer, som for eksempel filen \verb|break2.c|, ved å bruke en \verb|<fil>:<linje eller funksjon>|-syntaks:

\begin{lstlisting}[mathescape=true,keywordstyle=\color{black}]  
(gdb) break break2.c:library_call
Breakpoint 4 at 0x55555555468a: file source/break2.c, line 3.
\end{lstlisting}


Om man kun trenger å stoppe et sted en gang, så har GDB også midlertidige \textit{breakpoints}, som på samme måte som vanlige \textit{breakpoints}, men med kodeordet \verb|tbreak| istedenfor \verb|break|:

\begin{lstlisting}[mathescape=true,keywordstyle=\color{black}]
(gdb) tbreak library_call
Note: breakpoint 4 also set at pc 0x55555555468a.
Temporary breakpoint 5 at 0x55555555468a: file source/break2.c, line 3
\end{lstlisting}


Om dere nå kaller \verb|info break| vil dere se at \verb|Disp|-feltet (disposition) er forskjellig fra vanlige \textit{breakpoints}:

\begin{lstlisting}[mathescape=true,keywordstyle=\color{black}]
4 breakpoint keep y 0x000055555555468a in library_call at
->source/break2.c:3
5 breakpoint del  y 0x000055555555468a in library_call at
->source/break2.c:3
\end{lstlisting}


hvor \verb|keep| betyr at \textit{breakpointet} vil beholdes, mens \verb|del| indikerer at et \textit{breakpoint} vil slettes etter at det blir truffet.

En siste ting som er greit å vite om \textit{breakpoints} er at du kan tilknytte vilkårlige kommandoer til hvert enkelt \textit{breakpoint}, som vil kjøre hver gang \textit{breakpointet} aktiveres. Om dere for eksempel ønsker å legge inn en \verb|printf|, uten å måtte endre koden og gjenkompilere, kan dere gjøre slik:

\begin{lstlisting}[mathescape=true,keywordstyle=\color{black}]
(gdb) break 14
Breakpoint 6 at 0x555555554665: file source/break.c, line 14.
(gdb) command 6
Type commands for breakpoint(s) 6, one per line.
End with a line saying just "end".
>printf "Value of i is %d\n", i
>end
\end{lstlisting}

Dette vil skrive ut verdien til \verb|i| på samme måte som `gammeldags \verb|printf|-debugging' gjør, men du slipper å tukle med koden og kompilere på nytt. GDB kan faktisk kalle en hvilken som helst funksjon - også bibliotek-funksjoner - via \textit{breakpoint}-kommandoer. Dette er en veldig kraftig egenskap.

\subsection{Inspeksjon og endring av variabler}

GDB kan inspisere og endre variabler under kjøring. For å illustrere hvordan dette kan gjøres, skal vi bruke programmet \verb|inspect|. Kall \verb|make inspect|, og deretter \verb|gdb -tui inspect| for å starte GDB.

Sett opp et break-point i slutten av \verb|main| ved å kalle \verb|break 21|. Kall deretter \verb|run|. Når GDB stopper vil det være like før programmet returnerer. Kall nå \verb|print stat_array| og \verb|print dyn_array|:

\begin{lstlisting}[mathescape=true,keywordstyle=\color{black}]
Breakpoint 1, main () at source/inspect.c:21
(gdb) print stat_array
$\texttt{\$}$1 = {1, 2, 3, 4, 5}
(gdb) print dyn_array
$\texttt{\$}$2 = (int *) 0x555555756260

\end{lstlisting}




Som man kan se ut fra koden i \verb|inspect.c|, inneholder \verb|stat_array| og \verb|dyn_array| den samme informasjonen, men de er allokert henholdsvis statisk og dynamisk. Når vi printer den dynamiske tabellen, får vi bare pekeren. Dette skjer fordi GDB ikke kan være sikker på nøyaktig hva som ligger bak pekeren. Ettersom vi vet hva pekeren peker til, kan vi utnytte en spesiell syntaks for å skrive den ut som et array:

\begin{lstlisting}[mathescape=true,keywordstyle=\color{black}]
(gdb) print *dyn_array@5
$\texttt{\$}$3 = {1, 2, 3, 4, 5}

\end{lstlisting}


Dette betyr bare at vi skal dereferere pekeren, og ta 5 elementer, og skrive dem ut som en tabell. Det er ikke veldig ofte man trenger denne syntaksen, men den kan være veldig nyttig i noen sammenhenger.

I tillegg, kan GDB også skrive ut structs:

\begin{lstlisting}[mathescape=true,keywordstyle=\color{black}]
(gdb) print PID
$\texttt{\$}$4 = {kp = 100, ki = 0, kd = 0}
\end{lstlisting}

Kanskje det mest nyttigste med GDB er at den kan endre variabler underveis ved å bruke kommandoen \verb|set|:

\begin{lstlisting}[mathescape=true,keywordstyle=\color{black}]
(gdb) print some_number
$\texttt{\$}$5 = 1
(gdb) set some_number = 2
(gdb) print some_number
$\texttt{\$}$6 = 2
\end{lstlisting}


hvor \verb|$|-tegnet brukes som en referanse til GDB sin variabelhistorie. Dette tallet vil øke med en for hver kommando.

\subsubsection*{Display}

GDB har også en kommando kalt \verb|display|, som fungerer på samme måte som \verb|print|, men automatisk skriver ut en variabel hver gang et \textit{breakpoint} aktiveres, så lenge variabelen finnes på øverste stack-frame. For å stoppe en aktiv \verb|display| brukes kommandoen \verb|undisplay| med ID-en til \verb|display|-et.

For å prøve ut \verb|display| kan dere kalle \verb|break 15| for å opprette et \textit{breakpoint} ved linja \verb|dyn_array[i] = i + 1;|, etterfulgt av \verb|display *dyn_array@5|, og til slutt \verb|run| for å starte programmet på nytt. Hver gang dere treffer break-pointet vil dere se at ett nytt element er blitt satt i \verb|dyn_array|. For å slippe å skrive \verb|continue| mange ganger kan dere forresten bare trykke enter for å repetere siste kommando som ble kjørt.

\subsection{\texttt{step} og \texttt{next}}

Det som gjør GDB spesielt nyttig er at den lar brukeren gå gjennom koden linje for linje. Kompiler og debug programmet \verb|step| ved å kalle \verb|make step| etterfulgt av \verb|gdb -tui step|. I GDB kaller dere \verb|break main|. fulgt av \verb|run|. Programmet vil nå stoppe på linjen \verb|int prod1 = multiply(2, 6);|. Kall nå kommandoen \verb|step| fire-fem ganger og legg merke til hva som skjer.

Deretter starter dere programmet på nytt ved å kalle \verb|run| igjen. Denne gangen kaller dere \verb|next| isdenfor \verb|step| et par ganger. Som dere ser, fungerer de to kommandoene litt annerledes:

\begin{itemize}
    \item \verb|step| vi gå en kodelinje videre, og for hvert funksjonskall den møter på, vil \verb|step| gå inn i implementasjonen.
    \item \verb|next| vil gjøre det samme som \verb|step|, men vil kjøre alle funksjonskall den møter i bakgrunnen - uten å gå inn i implementasjonen.
\end{itemize}

Altså: når du kaller \verb|step| vil den kun gå inn i en implementasjon dersom det finnes et symbol for funksjonen. Dermed vil den gå inn i all kode kompilert med flagget \verb|-g|, men hoppe over ting den ikke kan lese. Dermed vil ikke \verb|step| prøve å gå inn i implementasjonen til for eksempel \verb|printf|.

\subsection{\texttt{until} og \texttt{finish}}

Om man kommer inn i en løkke, eller havner i en funksjon som man er sikker på er feilfri, så er det mulig å kunne hoppe ut av den. For dette formålet har GDB kommandoene \verb|until| og \verb|finish|. Kall først \verb|make until|, og deretter \verb|gdb -tui until|. Inne i GDB oppretter dere et \textit{breakpoint} for \verb|main| ved å kalle \verb|break main|.

Hvis dere nå kaller \verb|run| vi dere stoppe i toppen av \verb|for|-løkka, fordi dette er den første interessante kodelinjen i \verb|main|. Dersom man nå ønsker å hoppe over løkken, og fortsette rett etter den, kan man kalle \verb|until| fire ganger. Dere vil se at GDB nå stopper på linje 21 som er rett før \verb|pointless_function|-kallet.

Kommandoen \verb|until| skal hoppe til kodelinjen som er 1 større enn nåværende kodelinje. Grunnen til at dere måtte kalle \verb|until| tre ganger istedenfor en, er at GDB arbeider med maskinkoden til programmet. I C-kode, ser det ut som om at vi gjør en test i starten av løkken - men i den ferdigkompilerte koden skjer dette faktisk i bunnen av løkken. Dette er ikke veldig viktig å huske på, men det er greit å vite dersom en kommando tilsynelatende oppfører seg litt rart.

Når dere så har kommet til \verb|pointless_function|, kaller dere \verb|step| for å gå inn i implementasjonen. Sett nå at dere har gjort dere ferdig inne i denne funksjonen og ønsker å komme tilbake til \verb|main|. Man kunne brukt midlertidige \textit{breakpoints} og \verb|continue|, men dette blir fort tungvint. Man kan heller bruke \verb|finish| som er GDB sin egendefinerte kommando for akkurat dette. Denne kommandoen vil fullføre et funksjonskall, og stoppe kodeutføringen rett etter funksjonen returnerer.


\subsection{Callstack}


Ofte er tilfellet at en variabel uventet skifter verdi, eller at en funksjon kalles ut av det blå. Dette skjer vanligvis fordi man har en lang linje av funksjoner etter hverandre, der en av dem er feil. For å gjøre det lettere å inspisere hva som faktisk skjer, kan GDB skrive ut kall-stacken, eller hver enkelt \textit{stack-frame}.



Kall først \verb|make trace| for å kompilere programmet \verb|trace|, etterfulgt av \verb|gdb -tui| \verb|trace| for å debugge det. Fra GDB kan dere nå kalle \verb|watch global_value| for å stoppe når den globale variabelen \verb|global_value| på mystisk vis skifter verdi. Om dere nå kaller \verb|run|, vil dere få en output som ligner på denne:


\begin{lstlisting}[mathescape=true,keywordstyle=\color{black}]
(gdb) watch global_value
Hardware watchpoint 1: global_value
(gdb) run
Starting program: /home/student/Desktop/Øving - 4/programs/trace

Hardware watchpoint 1: global_value

Old value = 0
New value = 1
myst_func_1 (level=0) at source/trace.c:40

\end{lstlisting}


Variabelen skifter altså verdi fra 0 til 1, og i dette tilfellet skjedde det før linje 40. Som man kan se, så kan vi ikke være helt sikre på nøyaktig hvilken linje endringen skjedde, ettersom GDB egentlig jobber med maskinkoden til programmet, og prøver deretter så godt den kan å koble maskinkoden til C-koden. Heldigvis er det ikke så veldig interessant hvilken linje endringen skjedde på, så lenge vi får en anelse. 

Kall nå \verb|backtrace|. Dette vil skrive ut kall-stacken til programmet - altså hvilke funksjonskall som har hendt til nå. Dere vil få noe som ser slik ut:

\begin{lstlisting}[mathescape=true,keywordstyle=\color{black}]
(gdb) backtrace
#0 myst_func_1 (level=0) at trace.c:40
#1 0x0000555555554929 in myst_func_4 (level=1) at source/trace.c:96
#2 0x0000555555554956 in myst_func_4 (level=2) at source/trace.c:105
#3 0x0000555555554956 in myst_func_4 (level=3) at source/trace.c:105
#4 0x000055555555483c in myst_func_2 (level=4) at source/trace.c:59
#5 0x000055555555481e in myst_func_2 (level=5) at source/trace.c:53
#6 0x000055555555481e in myst_func_2 (level=6) at source/trace.c:53
#7 0x0000555555554938 in myst_func_4 (level=7) at source/trace.c:99
#8 0x0000555555554956 in myst_func_4 (level=8) at source/trace.c:105
#9 0x000055555555483c in myst_func_2 (level=9) at source/trace.c:59
#10 0x0000555555554791 in myst_func_1 (level=10) at source/trace.c:30
#11 0x0000555555554979 in main () at source/trace.c:113

\end{lstlisting}

Helt på bunnen vil dere se at ugangspunktet vårt var \verb|main|, som så kalte \verb|myst_func1| med \verb|level|-parameter lik 10. Deretter vet vi at en rekke kall til forskjellige \verb|myst_func|-funksjoner blir gjort, før vi kommer til \verb|myst_func_1| som må ha endret \verb|global_value|. En slik utskrift av kall-stacken kan gi en veldig god indikasjon på hvor ting går galt.


Om man har lyt på en spesifikk \textit{stack-frame} kan GDB også gi oss det. En stackframe er ett nivå på kall-stacken, og inneholder informasjon om hvilken funksjon som ble kalt, og hvilke argumenter som ble gitt. Kall \verb|frame| for å skrive ut øverste stack-frame, eller for eksempel \verb|frame 8| for å skrive ut en stack-frame 8 nivåer ned:

\begin{lstlisting}[mathescape=true,keywordstyle=\color{black}]
(gdb) frame 6
#6 0x000055555555481e in myst_func_2 (level=6) at source/trace.c:53
(gdb) frame 8
#8 0x0000555555554956 in myst_func_4 (level=8) at source/trace.c:105
\end{lstlisting}


\subsection{Segfault}

For å illustrere en segmentation fault, kan dere kompiler programmet \verb|seg1| ved å kalle \verb|make seg1|, etterfulgt av \verb|./seg1|. Dette programmet vil prøve å allokere minne den ikke eier eller har tilgang til. For å undersøke hva som er feil, bruker vi GDB som vanlig ved å kalle \verb|gdb -tui seg1|. Dersom vi først nå kjører programmet ved å kalle \verb|run| vil vi få noe slikt:

\begin{lstlisting}[mathescape=true,keywordstyle=\color{black}]
Program received signal SIGSEGV, Segmentation fault.
0x000055555555461e in main () at source/seg1.c:7
\end{lstlisting}


Dette forteller oss at vi gjorde en ulovlig minneoperasjon i \verb|seg1.c|, på linje 7. Vi har allerede bare fra dette mye å gå på dersom vi skal fikse på koden. Kall deretter \verb|print i|, for å skrive ut verdien av teller-variabelen. Den eksakte verdien er avhengig av arkitektur, men den burde være negativ. Utifra koden er det ganske åpenbart, men la oss nå late som om at vi fortsatt ikke er helt sikre på hvorfor.

For å få mer informasjon, kan vi kalle \verb|break 7|, for å sette opp et \textit{breakpoint} på linje 7. Så kaller dere \verb|info break| for å få en liste med informasjon om \textit{breakpointene} dere har i programmet. Dere vil ha noe som ser slik ut:

\begin{lstlisting}[mathescape=true,keywordstyle=\color{black}]
(gdb) break 7
Breakpoint 1 at 0x555555554607: file source/seg1.c, line 7.
(gdb) info break
Num    Type        Disp Enb  Address            What
1      breakpoint  keep y    0x0000555555554607 in main at source/seg1.c:7
\end{lstlisting}


Vi skal nå knytte kommandoer til \textit{breakpointet}, som skal kjøre hver gang vi kommer til det. Kall \verb|command 1| for å komme inn i kommando-modus for \textit{breakpoint} 1. Deretter skriver dere \verb|silent| på en linje, \verb|info local| på neste linje, og til slutt \verb|end| på siste linje - slik:

\begin{lstlisting}[mathescape=true,keywordstyle=\color{black}]
(gdb) command 1
Type commands for breakpoint(s) 1, one per line.
End with a line saying just "end".
>silent
>info local
>end
\end{lstlisting}


Om dere nå kaller \verb|info break| igjen vil dere se at \textit{breakpoint} 1 har fått to kommandoer knyttet til seg:

\begin{lstlisting}[mathescape=true,keywordstyle=\color{black}]
(gdb) info break
Num Type       Disp Enb  Address            What
1   breakpoint keep y    0x0000555555554607 in main at source/seg1.c:7
    silent
    info local
\end{lstlisting}



Her betyr \verb|silent| at GDB ikke vil skrive ut informasjon om \textit{breakpointet} hver gang det aktiverer, og \verb|info local| betyr at vi ønsker å skrive ut de lokale variablene som finnes til skjermen for inspeksjon. 

Om vi nå starter programmet på nytt ved å kalle \verb|run|, vil GDB stoppe på linje 7, og skrive ut \verb|i = 0|. Om man deretter repetitivt kaller \verb|continue| for å gå til neste løkke-iterasjon får man noe slikt:

\begin{lstlisting}[mathescape=true,keywordstyle=\color{black}]
i = 0
(gdb) continue
Continuing.
i = -1
Continuing.
i = -2
Continuing.
i = -3
Continuing.
i = -4

\end{lstlisting}


Det er nå åpenbart hva som foregår - vi teller nedover, når vi skulle ha telt oppover. Dette kommer av at vi har skrevet \verb|i--| i løkken, mens vi egentlig mente \verb|i++|.

\subsubsection{Tilleggs-kommentarer til Segfault}

Noen ganger så er man så uheldige at man ikke får generert en segfault til tross for at det er ting som man ikke burde gjort med minnet. Et eksempel får man ved å kompilere programmet \verb|seg2|. Om man kjører \verb|seg2|, med \verb|make seg2| og \verb|./seg2| får vi et program som allokerer ved \textit{compile time} et array av heltall, med størrelse 200. Programmet vil deretter prøve å skrive over 240 av disse elementene. Altså vil programmet prøve å skrive til 40 minnesegmenter som programmet egentlig ikke eier eller har allokert. 

I utgangspunktet burde dette ha gitt en segfault. Om dere ikke har fått en segfault før nå, så kan dere skrive \verb|Exit normally|. Refer til appendiks \ref{app:segfault} for å få en forklaring på hvorfor akkurat denne koden ikke gir en segfault.

\subsection{Minnelekkasje}

I teorien skjer ikke minnelekkasje dersom man er forsiktig. Men om man har et program som glemmer å frigjøre minne, og det kjører over lang nok tid - så vil programmet bryte sammen til slutt. Et av de mest effektive verktøyene for å bekjempe minnelekkasje er Valgrind. Valgrind emulerer en virtuell CPU og kjører programmet på denne. Dette gir brukeren full kontroll over hva som allokeres av minne - og hva som gis eller ikke gis tilbake.

Dersom man kaller \verb|make leak| fra kommandolinjen for å kompilere programmet \verb|leak|, etterfulgt av \verb|./leak 12| vil programmet skrive ut de 12 første Fibonacci-tallene. Hvis man nå kjører kommandoen \verb|valgrind --leak-check=yes ./leak 12|, vil man få en output som minner om denne:

\begin{lstlisting}[mathescape=true,keywordstyle=\color{black}]
==5504==
==5504== HEAP SUMMARY:
==5504==     in use at exit: 48 bytes in 1 blocks
==5504==   total heap usage: 2 allocs, 1 frees, 1,072 bytes allocated
==5504==
==5504== 48 bytes in 1 blocks are definitely lost in loss record 1 of 1
==5504==    at 0x4C31B0F: malloc (vg_replace_malloc.c:299)
==5504==    by 0x108345: create_array (leak.c:5)
==5504==    by 0x10880A: main (leak.c:23)
==5504==
==5504== LEAK SUMMARY:
==5504==    definitely lost: 48 bytes in 1 blocks
==5504==    indirectly lost: 0 bytes in 0 blocks
==5504==      possibly lost: 0 bytes in 0 blocks
==5504==    still reachable: 0 bytes in 0 blocks
==5504==         suppressed: 0 bytes in 0 blocks
==5504==
==5504== For counts of detected and suppressed errors, rerun with: -v
==5504== ERROR SUMMARY: 1 errors from 1 contexts (suppressed: 0 from 0)

\end{lstlisting}



Dette er en oppsummering av \verb|heap|-forbruk i løpet av programmets levetid. Som man kan se, har vi mistet 48 bytes fordi vi ikke frigjør allokeringen vi gjør. Valgrind hjelper oss også med å se hvor lekkasjen skjer; vi kan se av \textit{stack-tracen} at vi fra \verb|main| kaller \verb|create_array|, som deretter kaller \verb|malloc| uten at \verb|free| kalles noe sted.

